\documentclass{article}
\usepackage[utf8]{inputenc}
\usepackage{amssymb}
\usepackage{setspace}
\usepackage{amsmath}

\title{The first problem}
\author{Benjamin Li}
\date{March 2019}

\begin{document}

\setstretch{1.523}

\maketitle

\section{(a) and (b)}
Every positive number can be expressed in this fashion in exactly 2 ways\\
As a result, every negative number can also be expressed in this fashion in exactly 2 ways\\
0 cannot be expressed in this fashion (if an expression starts with $2^{i_0}$, then $2^{i_0+1}\nmid 0$)\\
This gives an answer to both (1) and (2)\\[22pt]
Claim $(P)$: moreover, for a positive number $n$, one way starts with a positive number, the other starts with a negative number\\
Define $k_n=v_2(n)$\\
Suppose we write $n=\sum\limits_{i=0}^{M} v_i 2^i$ in this fashion\\
Then $v_0=v_1=...=v_{k_n-1}=0,v_{k_n}\ne 0$ (It's clear)\\
Lemma 1:\\
(a)$2^{k_n}\le n$\\
(b)If $n<2^N$, then $n+2^{k_n}\le 2^N$\\
Prove: (a) is clear, for (b), just apply (a) to $2^N-n$\\
Since $2^{k_n}\le n<2^N\Rightarrow k_n<N$, so $k_{2^N-n}=k_n$\\
Q.E.D.\\
Lemma 2: $(P)$ is true for all $n=2^{n_0}$\\
Prove: It's clear that the only expressions of this type is $2^{n_0}$ and $-2^{n_0}+2^{n_0+1}$, so $(P)$ is true\\
Q.E.D.\\
We prove $(P)$ by showing that $(P)$ is true for every positive $n\le 2^N$, given arbitrary large $N$\\
We denote this statement by $(P_N)$\\
We prove $(P_N)$ by using induction on $N-k_n$\\
If $N-k_n=0$, then $n=2^N$, from lemma 2 we see $(P_N)$ is true\\
Suppose $(P_N)$ is true for all $N-k_n\le l$\\
For $N-k_n=l+1$\\
If $n=2^{k_n}$ then from lemma 2 $(P_N)$ is true\\
Otherwise $0<n-2^{k_n}<n+2^{k_n}\le 2^N$\\
We have both $N-k_{n-2^{k_n}}$ and $N-k_{n+2^{k_n}}$ are less than $N-k_n$\\
So by our assumption we have\\
$n-2^{k_n}$ can be expressed in this fashion with negative starting number in exactly 1 way\\
$n+2^{k_n}$ can be expressed in this fashion with positive starting number in exactly 1 way\\
By doing this we find that there's exactly 1 way to express $n$ in this fashion with positive starting number\\
And there's exactly 1 way to express $n$ in this fashion with negative starting number\\
Q.E.D.
\section{(c)}
We prove that every positive number $n$ can be expressed as an alternating sum of an increasing sequence of Fibonacci numbers $F_i$ ($F_1=F_2=1$)\\
And obviously for some $n$ there are multiple ways\\
For example $5=-8+13=1-2+3-5+8$\\[22pt]
Lemma: $\forall n\in \mathbb{Z}^+,\exists u_i\in \{0,1\}$ such that\\
$n=\sum\limits_{i=2}^N u_i F_i$\\
Prove: use induction\\
For $n=1$, it is clear\\
For $n>1$, pick the largest $F_{i_0}$ that is $\le n$\\
Then $n-F_{i_0}<F_{i_0+1}-F_{i_0}=F_{i_0-1}<F_{i_0}$\\
And by induction it is true for $n-F_{i_0}$\\
So it is true for $n$\\
Q.E.D.\\
For convenience, denote $u_1=u_{N+1}=0$\\
Pick $v_i=u_{i-2}-u_{i-1}\in \{-1,0,1\}(i=3,4,...,N+2)$\\
Claim: $n=\Sigma\triangleq\sum\limits_{i=3}^{N+2} v_i F_i$\\
Prove: $\Sigma=\sum\limits_{i=3}^{N+2}(u_{i-2}-u_{i-1})F_i$\\
$=2u_1-u_{N+1}F_{N+2}+\sum\limits_{i=2}^N u_i(F_{i+2}-F_{i+1})=\sum\limits_{i=2}^N u_i F_i=n$\\
And note that $|\sum\limits_{i=a}^b v_i|=|u_{a-2}-u_{b-1}|\le 1$\\
So there can't be consecutive subsequence of $v_i$ that goes like\\
$1,0,0,...,0,1$ or\\
$-1,0,0,...,0,-1$

\end{document}
